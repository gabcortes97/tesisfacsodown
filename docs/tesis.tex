% This is the Reed College LaTeX thesis template. Most of the work
% for the document class was done by Sam Noble (SN), as well as this
% template. Later comments etc. by Ben Salzberg (BTS). Additional
% restructuring and APA support by Jess Youngberg (JY).
% Your comments and suggestions are more than welcome; please email
% them to cus@reed.edu
%
% See https://www.reed.edu/cis/help/LaTeX/index.html for help. There are a
% great bunch of help pages there, with notes on
% getting started, bibtex, etc. Go there and read it if you're not
% already familiar with LaTeX.
%
% Any line that starts with a percent symbol is a comment.
% They won't show up in the document, and are useful for notes
% to yourself and explaining commands.
% Commenting also removes a line from the document;
% very handy for troubleshooting problems. -BTS

% As far as I know, this follows the requirements laid out in
% the 2002-2003 Senior Handbook. Ask a librarian to check the
% document before binding. -SN

%%
%% Preamble
%%
% \documentclass{<something>} must begin each LaTeX document
\documentclass[12pt,twoside]{templates/facsothesis}
\usepackage{helvet}
\renewcommand{\familydefault}{\sfdefault}

% Packages are extensions to the basic LaTeX functions. Whatever you
% want to typeset, there is probably a package out there for it.
% Chemistry (chemtex), screenplays, you name it.
% Check out CTAN to see: https://www.ctan.org/
%%
\ifxetex
  \usepackage{polyglossia}
  \setmainlanguage{spanish}
  % Tabla en lugar de cuadro
  \gappto\captionsspanish{\renewcommand{\tablename}{Tabla}
          \renewcommand{\listtablename}{Índice de tablas}}
\else
  \usepackage[spanish,es-tabla]{babel}
\fi
%\usepackage[spanish]{babel}
\usepackage{graphicx,latexsym}
\usepackage{amsmath}
\usepackage{amssymb,amsthm}
\usepackage{longtable,booktabs,setspace}
\usepackage{chemarr} %% Useful for one reaction arrow, useless if you're not a chem major
\usepackage[hyphens]{url}
% Added by CII
%\usepackage{hyperref}
\usepackage[colorlinks = true,
            linkcolor = blue,
            urlcolor  = blue,
            citecolor = blue,
            anchorcolor = blue]{hyperref}
\usepackage{lmodern}
\usepackage{titlesec}
\titleformat{\chapter}[display]{\normalfont\bfseries}{}{0pt}{\Huge}
\titlespacing*{\chapter}{0pt}{-50pt}{40pt}
\usepackage{float}
\floatplacement{figure}{H}
% End of CII addition
\usepackage{rotating}
\usepackage{placeins} % para fijar la posición de las tablas con \FloatBarrier


\usepackage[]{natbib}


% Next line commented out by CII
%\usepackage{biblatex}
%\usepackage{natbib}
% Comment out the natbib line above and uncomment the following two lines to use the new
% biblatex-chicago style, for Chicago A. Also make some changes at the end where the
% bibliography is included.
%\usepackage{biblatex-chicago}
%\bibliography{thesis}


% Added by CII (Thanks, Hadley!)
% Use ref for internal links
\renewcommand{\hyperref}[2][???]{\autoref{#1}}
\def\chapterautorefname{Chapter}
\def\sectionautorefname{Section}
\def\subsectionautorefname{Subsection}
% End of CII addition

% Added by CII
\usepackage{caption}
\captionsetup{width=5in}
% End of CII addition

% \usepackage{times} % other fonts are available like times, bookman, charter, palatino

% Syntax highlighting #22

% To pass between YAML and LaTeX the dollar signs are added by CII
\title{PERFILES DE INDIVIDUALISMO Y SU RELACIÓN CON EL APOYO A LA DEMOCRACIA DELEGATIVA EN LA SOCIEDAD CHILENA}
\author{GABRIEL CORTÉS PAREDES}
% The month and year that you submit your FINAL draft TO THE LIBRARY (May or December)
\date{Santiago de Chile, 2023}
\division{}
\advisor{Profesora guía: Macarena Orchard}
\institution{FACULTAD DE CIENCIAS SOCIALES E HISTORIA}
\degree{Tesis para optar al grado de magíster en Métodos para la Investigación Social}
%If you have two advisors for some reason, you can use the following
% Uncommented out by CII
% End of CII addition

%%% Remember to use the correct department!
\department{}
% if you're writing a thesis in an interdisciplinary major,
% uncomment the line below and change the text as appropriate.
% check the Senior Handbook if unsure.
%\thedivisionof{The Established Interdisciplinary Committee for}
% if you want the approval page to say "Approved for the Committee",
% uncomment the next line
%\approvedforthe{Committee}

% Added by CII
%%% Copied from knitr
%% maxwidth is the original width if it's less than linewidth
%% otherwise use linewidth (to make sure the graphics do not exceed the margin)
\makeatletter
\def\maxwidth{ %
  \ifdim\Gin@nat@width>\linewidth
    \linewidth
  \else
    \Gin@nat@width
  \fi
}
\makeatother

%Added by @MyKo101, code provided by @GerbrichFerdinands

\setlength\parindent{0pt}


% Added by CII

\providecommand{\tightlist}{%
  \setlength{\itemsep}{0pt}\setlength{\parskip}{0pt}}

\Acknowledgements{

}

\Dedication{

}

\Preface{

}

\Abstract{

}

	\usepackage{booktabs}
\usepackage{longtable}
\usepackage{array}
\usepackage{multirow}
\usepackage{wrapfig}
\usepackage{float}
\usepackage{colortbl}
\usepackage{pdflscape}
\usepackage{tabu}
\usepackage{threeparttable}
\usepackage{threeparttablex}
\usepackage[normalem]{ulem}
\usepackage{makecell}
\usepackage{xcolor}
% End of CII addition
%%
%% End Preamble
%%
%
\let\chaptername\relax
\begin{document}
\bibliographystyle{apa-good}
% Everything below added by CII
  \maketitle

\frontmatter % this stuff will be roman-numbered
\pagestyle{empty} % this removes page numbers from the frontmatter



%  \hypersetup{linkcolor=black}
  \setcounter{tocdepth}{1}
  \setlength{\parskip}{0pt}
  \tableofcontents

\setlength\parskip{1em plus 0.1em minus 0.2em}

  \listoftables

  \listoffigures



\mainmatter % here the regular arabic numbering starts
\titleformat{\chapter}{\normalfont\Huge\bfseries}{\thechapter}{1em}{}
\pagestyle{fancyplain} % turns page numbering back on

\hypertarget{prefacio}{%
\chapter*{Prefacio}\label{prefacio}}
\addcontentsline{toc}{chapter}{Prefacio}

\emph{``A toast, Jedebiah, to love on my terms. Those are the only terms anybody ever knows -- his own''} (Orson Welles, 1941)

\emph{``If success and failure are the result of individual effort, those at the top can hardly be blamed -- unless, of course, they are politician''} (Bellah et al, 1996, p.xv)

\hypertarget{resumen}{%
\chapter*{Resumen}\label{resumen}}
\addcontentsline{toc}{chapter}{Resumen}

\hypertarget{agradecimientos}{%
\chapter*{Agradecimientos}\label{agradecimientos}}
\addcontentsline{toc}{chapter}{Agradecimientos}

Agradecimientos aquí.

\hypertarget{antecedentes}{%
\chapter{Antecedentes}\label{antecedentes}}

Si bien se ha observado que el apoyo a la democracia en Chile ha sido históricamente alto \citep{navia2019}, se debe notar con atención algunos síntomas que indican una caída en este soporte, particularmente durante este último año \citep{cep}, acompañada de un fortalecimiento de la imagen pública de Augusto Pinochet y su dictadura \citep{cadem2023, cerc-mori}, así como de los últimos resultados electorales que han sido favorables para José Antonio Kast y su Partido Republicano, quienes han defendido abiertamente el legado del régimen militar y que han sido descritos como populistas radicales de derecha \citep{diaz2023}.

Esto se da, además, en el contexto de un crisis de representatividad que se ha profundizado durante la última década, pero cuyos orígenes se pueden retrotraer incluso hacia fines de los años 90, en los primeros años de la transición democrática chilena tras el fin de la Dictadura Militar en 1990 \citep{luna2016}. Este período de la historia chilena fue tempranamente considerado como exitoso, debido a su rápida consolidación institucional y económica, particularmente en comparación a procesos similares en otros países de América Latina.

En el resto de la región, por el contrario, se observaban dificultades en los procesos de consolidación de los nuevos regímenes, que fueron descritos por Guillermo O'Donnell \citeyearpar{odonnell1994} bajo el concepto de \emph{democracia delegativa}. Esta variante de democracia se caracterizaría por la presencia de un presidente que es invetido de un liderazgo fuerte que le permite pasar por sobre el control de otras instituciones del Estado con el fin de sanar y unir a la nación. En otros palabras, una democracia delegativa cuenta con una fuerte responsabilidad vertical (o \emph{vertical accountability}), es decir, hacia el pueblo o los ciudadanos, pero no hacia otros poderes o instituciones (\emph{horizontal accouintability}) \citep{odonnell1994}.

Esta descripción, por cierto, no se acomoda a la realidad chilena. Tampoco, sin embargo, se podría decir que Chile es plenamente una democracia delegativa donde convivan ambas formas de rendición de cuentas. Por el contrario, lo que se observa es que la contundencia del control horizontal entre instituciones es acompañada por un profundo desarraigo entre las élites políticas y la ciudadanía \citep{luna2016}.

En este contexto, si bien Chile no es una democracia delegativa, no sería raro pensar que aparezcan tendencias que apelen a una mayor rendición de cuentas vertical, incluso a expensas de debilitar las instituciones de control horizontal, respaldando así soluciones autoritarias o no-democráticas \citep{carlin2018}

Por supuesto, la disminución del apoyo a la democracia y el surgimiento de opciones autoritarias o populistas no es un fenómeno únicamente local, y ha sido estudiado ampliamente en varias regiones del mundo bajo diversas etiquetas, tales como \emph{liderazgos fuertes, no-democráticos o delegativos} \citep{carlin2011, carlin2018, crimston2022, kang2018, lima2021, selvanathan2022, xuereb2021}, \emph{populismos} \citep{baro2022, gidron2020, nowakowski2021}, o \emph{derecha populista radical} \citep{diaz2023, donovan2019, donovan2021}. También se ha puesto esfuerzos en identificar sus determinantes, entre los que se pueden contar factores culturales \citep{lima2021, marchlewska2022, selvanathan2022}; factores económicos objetivos y subjetivos \citep{arikan2019, rico2020, wu2019, xuereb2021}; bajo bienestar o estatus subjetivo \citep{gidron2020, nowakowski2021}; sentimientos de anomia y de polarización moral \citep{crimston2022}; la pertenencia a una minoría étnica o religiosa con baja integración nacional \citep{eskelinen2020}; así como otros rasgos o valores personales \citep{baro2022, marchlewska2019, rico2020}.

Como se puede notar, pese a que el espectro Individualismo-Colectivismo se considera una de las más importantes y más estudiadas dimensiones de la cultura \citep{binder2019, fatehi2020}, y que ha sido utilizado como variable explicativa en diversos estudios sobre economía \citep{binder2019, kyriacou2016, germani2021, toikko2020}, capital social \citep{beilmann2018}; género \citep{dabiriyantehrani2022, davis2019}, familia \citep{al-hassan2021, rudy2006}, trabajo \citep{refslund2022, solis2018, stewart2020}, cumplimiento de normas \citep{varet2018, zhang2020}, y actitudes frente a la pandemia y la vacunación \citep{card2022}, su conexión con las preferencias políticas y el respaldo hacia distintos modelos de democracia aún ha sido escasamente explorada.

La literatura existente se ha preocupado más bien de explorar la relación entre individualismo, colectivismo y autoritarismo. Al respecto, se ha descrito que entre estudiantes universitarios estadounidenses la relación entre individualismo y colectivismo es en realidad ortogonal, ubicando al primero en el polo opuesto del autoritarismo \citep{gelfand1996}. En una serie de estudios comparativos en varios países, por otro lado, se ha complejizado esa relación, encontrando una relación positiva entre autoritarismo e individualismo vertical -- que privilegia la competencia y jerarquía entre individuos - pero no con el individualismo horizontal, que privilegia la competencia y la igualdad entre individuos \citep{kemmelmeier2003}. Se ha observado, además, que el individualismo vertical está asociado con orientaciones de dominancia social \citep{strunk1999} y con el voto conservador en los Estados Unidos \citep{zhang2009}. También se ha explorado la relación entre individualismo, colectivismo y autoritatismo a nivel familiar, encontrando que madres colectivistas apoyan un estilo más autoritario de parentalidad \citep{rudy2006}. Por otro lado, se ha argumentado que culturales individualistas promueven una mejor gobernanza, desincentivando la corrupción, el nepotismo y el clientelismo \citep{kyriacou2016}.

Estos estudios comparten limitaciones, como su escasez y dispersión en el tiempo, su carácter exploratorio, o el circunscribir las definiciones de individualismo y colectivismo a un nivel cultural, sin detenerse a analizar las posibles difracciones dentro de una misma sociedad. Además, ninguna de estas investigaciones ha explorado estos fenómenos en Chile o en América Latina. Tampoco parece haberse explorado la relación con el apoyo a una democracia delegativa que, si bien contiene rasgos autoritarios e iliberales, parece ser un fenómeno diferente \citep{carlin2011, carlin2018}. De tal modo, se buscará abordar esa brecvha incluyendo un giro en la conceptualización de individualismo, que busca dejar de entenderlo como una dimensión cultural para pasar a definirlo como el resultado de procesos sociohistóricos de individualización que difieren no solo entre culturas, sino también dentro de una misma sociedad \citep{martuccelli2018, silvapalacios2015}.

La individualización es una corriente sociohistórica que transforma las relaciones de los sujetos con la autoridad, así como los soportes y las modalidades que autorizan su ejercicio \citep{araujo2021}. Por ello, parece interesante indagar cómo diferentes variantes de individualismo --resultado de difracciones de los procesos de individualización -- se relacionan con la pérdida de legitimidad de modalidades democráticas de autoridad, privilegiando, por ejemplo, liderazgos más fuertes, eficientes \citep{araujo2022, araujo2022a}, o auténticos \citep{gauthier2021}.

En visto de todo lo planteado, se propone como pregunta de investigación la siguiente: \textbf{¿Cuál es la relación entre distintos perfiles de individualismo y el apoyo a una democracia delegativa en la sociedad chilena?}

Lo que se traduce al objetivo general de \textbf{Establecer la relación entre distintos perfiles de individualismo y el apoyo a una democracia delegativa en la sociedad chilena}. A sus vez, de aquí se desprenden los siguientes objetivos específicos:

\begin{itemize}
\tightlist
\item
  Identificar los perfiles de individualismo presentes en la sociedad chilena
\item
  Describir el apoyo a una democracia delegtiva en la sociedad chilena
\item
  Relacionar las variantes de individualismo con el apoyo a una democracia delegativa en la sociedad chilena
\end{itemize}

\hypertarget{marco-teuxf3rico}{%
\chapter{Marco Teórico}\label{marco-teuxf3rico}}

\hypertarget{democracia-delegativa}{%
\section{Democracia delegativa}\label{democracia-delegativa}}

El concepto de democracia delegativa fue acuñado por el sociólogo argentino Guillermo O'Donnell para describir la situación institucional de las nuevas democracias latinoamericanas surgidas tras el fin de los regímenes autoritarios en la región durante las décadas de 1980 y 1990. Esta forma de democracia se basa en la premisa de que el ganador de las elecciones presidenciales tiene derecho a gobernar sin restricciones, considerándose la encarnación del país y el principal defensor de sus intereses \citep{odonnell1994}. Se diferencian de las democracias representativas consolidadas en que una fuerte responsabilidad vertical (es decir, frente a sus electores) no es acompañada por una rendición de cuentas horizontal, esto es, hacia otras instituciones del Estado \citep{odonnell1994}.

Bajo esta definición, Chile se ha tendido a tomar como un contraejemplo, destacando la fuerza de sus instituciones democráticas surgidas tras el fin de la Dictadura \citep{odonnell1994, carlin2018}. Sin embargo, y como se ilustra en la tabla 1, en Chile la rendición de cuentas horizontal no es acompañada por una rendición de cuentas vertical, lo que se traduciría en una \emph{Uprooted democracy} marcada por una profunda crisis de representatividad \citep{odonnell1994}.

\begin{table}

\caption{\label{tab:unnamed-chunk-2}Comparación variantes democracia}
\centering
\begin{tabu} to \linewidth {>{\centering}X>{\centering}X>{\centering}X}
\toprule
\multicolumn{1}{c}{Tipo de democracia} & \multicolumn{1}{c}{Vertical Accountability} & \multicolumn{1}{c}{Horizontal Accountability}\\
\midrule
Democracia Representativa & + & +\\
Democracia Delegativa & + & -\\
Democracia Desarraigada & - & +\\
\bottomrule
\multicolumn{3}{l}{\rule{0pt}{1em}\textit{Nota.} Tabla basada en O'Donnell (1994) y en Luna (2016)}\\
\end{tabu}
\end{table}

Frente a esto, no resulta contradictorio que una democracia caracterizada por su fuerza institucional puedan surgir en la población actitudades de preferencias por este tipo de gobierno \citep{carlin2011, carlin2018}. Según Carlin \citeyearpar{carlin2018}, las personas que apoyan una democracia delegativa en Chile se caracterizan por apoyar a líderes fuertes que unan al país y lo guíen en tiempos de crissi, mostrar orientaciones no-liberales (\emph{iliberals}) hacia los derechos políticos y falta de compromiso hacia los derechos humanos. Sin embargo, y quizás paradójicamente, este perfil sigue prefiriendo la democracia sobre otras formas de gobierno.

Los liderazgos fuertes, de tal manera, se constituyen como una de las dimensiones fundamentales de las democracias delegativas. Subyace aquí la idea de una nación concebida como un ser orgánico, un verdadero Leviatán del que líder es su cabeza, y cuya función es ``sanar la nación uniendo sus fragmentos dispersos en un todo harmonioso'' \citep[pp.60]{odonnell1994}.

De lo anterior, se desprende un segunda característica esencial de esta vrariante de democracia. El líder, para cumplir su cometido, debe saber combinar elementos emocionales y carismáticos con otros altamente técnicos, precisamente bajo la justificación de ``sanar'' la nación \citep{odonnell1994}. Esta impronta tecnocrática mezclada con elementos emocionales no es del todo desconocidas en Chile, como se observaría en el tipo ideal portaliano \citep{araujo2013}, una forma sociohistórica de ejercicio de la autoridad en Chile. Por otro lado, podría también recordar a la discusión sobre el surgimiento de actitudes tecnocráticas y tecnopopulistas en países europeos y su relación, muchas veces contradictoria, con la democracia \citep{chiru2022, ganuza2020, pilet2023}.

\hypertarget{individualismo}{%
\section{Individualismo}\label{individualismo}}

\hypertarget{individidualismo-colectivismo-como-una-dimensiuxf3n-de-la-cultura}{%
\subsection{Individidualismo-Colectivismo como una dimensión de la cultura}\label{individidualismo-colectivismo-como-una-dimensiuxf3n-de-la-cultura}}

\begin{itemize}
\tightlist
\item
  La forma más común en que se ha abordado el fenómeno del individualismo viene de la psicología cultural y, particularmente, de la comparación entre culturas.
\item
  Se suele conceptualizar, de tal forma, en par junto al colectivismo. Habrían culturas - y la gran mayoría de las veces, culturas es igual a países - individualistas, y otras colectivistas.
\item
  Tanto el individualismo como el colectivismo son conceptos de larga tradición intelectual. Con todo, su uso actual tuvo un gran auge a parti de 1980 con los estudios de Hofstede sobre la cultura laboral en trabajadores de 39 países de la empresa IBM: (Oyserman, 2002)

  \begin{itemize}
  \tightlist
  \item
    Hofstede definió 4 grandes dimensiones de la cultura: ``distancia de poder'', ``masculinidad'', ``aversión a las incertidumbre'', e ``individualismo''. Está última ha sido la que más atención ha captado. (Brewer, 2007)
  \item
    Para Hofstede, Individualismo-Colectivismo, a nivel cultural, conforman los dos polos de un único espectro (Oyserman, 2002). Las sociedades individualistas se caracterizan por lazos poco estrechos entre individuos, y en de cada quien se espera que se haga cargo de si mismo y de su familia inmediata. Sociedades colectivistas, en tanto, se caracterizan porque sus miembros están integrados desde su nacimiento a grupos fuertemente cohesionados, que los protegen a lo largo de su vidas en cambio por lealtad incuestionada (Yoon, 200)
  \end{itemize}
\item
  A pesar de que el propio Hofstede advierte que estas definiciones aplican 1) a un nivel cultural, y no al individual; y 2) son procesos dinámicos en que las culturas pueden transformarse -- estás recomendaciones no siempre han sido seguidas por los investigadores que han retomado está definición (Oyserman, 2002).
\item
  Para hacerse cargo del fenómeno a nivel individual se han elaborado conceptualizaciones alternativas: Por ejemplo, la idea del \emph{self-construal}, o la idea de si las personas se conciben a sí mismas de forma independiente o interdependiente de sus grupos. Además, está propuesta se diferencia de la Hofstede en que es un constructo bidimensional. Ahora bien, pese a que se ha insistido que el \emph{self-construal} y el individualismo-colectivismo son fenómenos diferentes, su operacionalización muchas veces se intercepta. Por lo demás, se mantiene una interpretación más o menos explícita que relaciona una concepción independiente con culturas individualistas.
\item
  También se ha vuelto popular la conceptualización que distingue dimensiones horizontales y verticales de individualismo y colectivismo. El colectivismo horizontal se caracteriza por la igualdad y la cooperación entre miembros del grupo; el colectivismo vertical valora el deber hacia el grupo en función al lugar en la jerarquía e que su ubique la persona; El individualismo horizontal valora la autonomía y la unicidad; El individualismo vertical, en tanto, valora la competencia, el logro y la jerarquía entre individuos (Méndez, 2008). Pese a que Triandis parece estar pensando en una conceptualización para el fenómeno a nivel cultural (Méndez, 2008), nuevamente suele usarse este tipo de escalas para describir diferencias entre individuos.
\item
  De tal modo, el uso de individualismo-colectivismo ha sido criticado por su falta de claridad conceptual, calificandolo como un concepto \emph{catch-all}, que se usa por defecto para explicar diferencias culturales:

  \begin{itemize}
  \tightlist
  \item
    Se observa acá una dimensión normativa: El individualismo se ha entendido como una dimensión esencial de la cultura estadounidense. También se ha asociado a la modernidad y al desarrollo (Voronov; Wang). Individualismo, así, suele tener una connotación positiva, mientras que colectivismo una negativa (moemeka2018), sobretodo en Estados Unidos y otros países anglosajones. De ahí que no sea de extrañar que la diferencia de individualismo y colectivismo puedan recordar, por ejemplo, las distinciones sociológicas clásicas entre sociedad moderna y sociedad tradicional; sociedad y comunidad; solidaridad orgánica y solidaridad mecánica.
  \item
    La falta claridad conceptual se puede observar en el estudio de Oyserman, quien a través de un análisis de contenido a las escalas más utilizadas, concluye que individualismo puede referirse a has 6 cosas distintas (independencia, logros, competencia, unicidad, autoconocimiento y comunicación directa); mientras que colectivismo a otras 8 (relaciones, pertenencia, deber, armonía, búsqueda de consejos, contextualidad, jerarquía, grupos).
  \item
    Brewer va más allá, indicando que en realidad ni siquiera hay verdadera simetría en las formas en que individualismo y colectivismo están operacionalizados: Así, mientras que los ítems para medir el primero suelen referirse a la identida y la agencia de los individuos; el segundo se suele medir como un sistema de valores.
  \item
    También se ha puesto atención a la falta de claridad de quiénes son los colectivos del colectivismo, no haciendo una clara distinción entre grupo, colectivo y comunidad.

    \begin{itemize}
    \tightlist
    \item
      Moemeka ha planteado que más que hablar de colectivismo se debería hablar de comunalismo. Los colectivos se forman por elección, de ahí que no haya verdadera contradicción entre colectivismo e individualismo. Por ejemplo, sindicatos y otros movimientos sociales serían colectivos que tienen mayor presencia en sociedades individualistas.
    \item
      Para Brewer, las escalas analizadas suelen medir relaciones interpersonales, y propone distinguir está dimensión de la colectiva (que lo entiende a grupos enteros, religiosos, étnicos o nacionales)
    \item
      Este giro permitiría explicar las ``anomalías'' observadas en estos estudios:

      \begin{itemize}
      \tightlist
      \item
        A veces, los individualistas puede ser tanto o más colectivistas que los colectivistas (Oyserman)
      \item
        Muchas veces, los colectivistas actúan de manera individualista (Vornov)
      \item
        Chile es un claro ejemplo. El colectivismo en Chile es alto. Más alto que en Corea del Sur, incluso ¿Implica esto que somos una sociedad colectivista? Lo cierto es que los niveles de individualismo en Chile son más altos que en Estados Unidos y que en Noruega.
      \end{itemize}
    \end{itemize}
  \end{itemize}
\end{itemize}

\hypertarget{individualismo-desde-la-sociologuxeda-del-individuo}{%
\subsection{Individualismo desde la Sociología del Individuo}\label{individualismo-desde-la-sociologuxeda-del-individuo}}

\hypertarget{propuesta-teuxf3rica-una-reconstrucciuxf3n-analuxedtica-del-individualismo}{%
\subsection{Propuesta teórica: Una reconstrucción analítica del individualismo}\label{propuesta-teuxf3rica-una-reconstrucciuxf3n-analuxedtica-del-individualismo}}

\hypertarget{legitimidad-de-la-individualidad}{%
\subsubsection{Legitimidad de la Individualidad}\label{legitimidad-de-la-individualidad}}

\hypertarget{concepciones-y-autoconcepciones-del-individuo}{%
\subsubsection{Concepciones y autoconcepciones del individuo}\label{concepciones-y-autoconcepciones-del-individuo}}

\hypertarget{tipos-de-individuaciuxf3n}{%
\subsubsection{Tipos de individuación}\label{tipos-de-individuaciuxf3n}}

\hypertarget{metodologuxeda}{%
\chapter{Metodología}\label{metodologuxeda}}

\hypertarget{datos}{%
\section{Datos}\label{datos}}

\hypertarget{variables}{%
\section{Variables}\label{variables}}

\hypertarget{muxe9todos}{%
\section{Métodos}\label{muxe9todos}}

\hypertarget{anuxe1lisis}{%
\chapter{Análisis}\label{anuxe1lisis}}

\hypertarget{anuxe1lisis-descriptivo}{%
\section{Análisis descriptivo}\label{anuxe1lisis-descriptivo}}

\hypertarget{modelos}{%
\section{Modelos}\label{modelos}}

\hypertarget{conclusiones}{%
\chapter{Conclusiones}\label{conclusiones}}

% %%%%%%%%%%%%%%%%%%%%%%%%%%%%%%%%%%%%%%%%%%%%%%%%%
% %%% Bibliography                              %%%
% %%%%%%%%%%%%%%%%%%%%%%%%%%%%%%%%%%%%%%%%%%%%%%%%%
% \addtocontents{toc}{\vspace{.5\baselineskip}}
% \cleardoublepage
% \phantomsection
% \addcontentsline{toc}{chapter}{\protect\numberline{}{Bibliography}}
\bibliography{tesis}

%% All books from our library (SfS) are already in a BiBTeX file
%% (Assbib). You can use Assbib combined with your personal BiBTeX file:
%% \bibliography{Myreferences,Assbib}. Of course, this will only work on
%% the computers at SfS, unless you copy the Assbib file
%%  --> /u/sfs/bib/Assbib.bib



\end{document}
